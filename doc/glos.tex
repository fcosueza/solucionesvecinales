\newglossaryentry{CTA}{%
	name={CTA},
	description={Un CTA, del inglés Call To Action, es una imagen o frase con un botón que busca persuadir a los usuarios para que realicen una acción concreta.}
}

\newglossaryentry{login}{%
	name={login},
	description={Proceso mediante el que se controla el acceso individual a un sistema informático mediante el uso de credenciales.}
}

\newglossaryentry{logout}{%
	name={logout},
	description={Proceso mediante el que un usuario que tienen una sesión activa en un sistema informático puede cerrar ésta y salir del sistema.}
}

\newglossaryentry{menú hamburguesa}{%
	name={menú hamburguesa},
	description={En diseño web un menú hamburguesa es un un tipo de menú que tiene forma de hamburguesa, consistiendo en un icono rectangular con varias líneas horizontales en su interior. Este menú suele emplearse para ahorrar espacio, especialmente en dispositivos móviles.}
}

\newglossaryentry{SSL}{%
	name={SSL},
	description={El protocolo SSL ó Secure Socket Layer es un protocolo de cifrado para comunicaciones en red que proporciona un canal seguro entre dos computadores o dispositivos que se comunican a través de internet o una red local}.
}

\newglossaryentry{HTTP}{%
	name={HTTP},
	description={El protocolo HTTP o HyperText Transfer Protocol es un protocolo desarrollado por Tim Berners-Lee en 1989 y que permite la transferencia de archivos entre dos computadores o dispositivos y que es la base de comunicación en la World Wide Web}.
}

\newglossaryentry{idempotente}{%
	name={idempotente},
	description={Un método HTTP es idempotente si el effecto que se produce en el servidor es el mismo si se manda una sola solicitud que si se mandan varias solicitudes iguales}.
}

\newglossaryentry{API}{%
	name={API},
	description={}.
}





