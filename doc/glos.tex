\newglossaryentry{CTA}{%
	name={CTA},
	description={Un CTA, del inglés Call To Action, es una imagen o frase con un botón que busca persuadir a los usuarios para que realicen una acción concreta.}
}

\newglossaryentry{login}{%
	name={login},
	description={Proceso mediante el que se controla el acceso individual a un sistema informático mediante el uso de credenciales.}
}

\newglossaryentry{logout}{%
	name={logout},
	description={Proceso mediante el que un usuario que tienen una sesión activa en un sistema informático puede cerrar ésta y salir del sistema.}
}

\newglossaryentry{menú hamburguesa}{%
	name={menú hamburguesa},
	description={Es un tipo de menú web que consiste en un icono rectangular con varias líneas horizontales en su interior, y que se despliega al ser clickado}
}

\newglossaryentry{SSL}{%
	name={SSL},
	description={El protocolo SSL ó Secure Socket Layer es un protocolo de cifrado para comunicaciones en red que proporciona un canal seguro entre dos computadores o dispositivos que se comunican a través de internet o una red local}.
}

\newglossaryentry{HTTP}{%
	name={HTTP},
	description={El protocolo HTTP o HyperText Transfer Protocol es un protocolo desarrollado por Tim Berners-Lee en 1989 y que permite la transferencia de archivos entre dos computadores o dispositivos y que es la base de comunicación en la World Wide Web}.
}

\newglossaryentry{idempotente}{%
	name={idempotente},
	description={Un método HTTP es idempotente si el effecto que se produce en el servidor es el mismo si se manda una sola solicitud que si se mandan varias solicitudes iguales}.
}

\newglossaryentry{API}{%
	name={API},
	description={Una API o Applicacition Programming Interface, es una pieza de código que permite a dos aplicaciones comunicarse entre sí para compartir información y funcionalidades}.
}

\newglossaryentry{frontend}{%
	name={frontend},
	description={En desarrollo web, el frontend es la parte de la aplicación que interactura con los usuarios, incluyendo todos los aspectos gráficos de la web y otros elementos que permiten navegar a los usuario dentro de la página}.
}

\newglossaryentry{backend}{%
	name={backend},
	description={En desarrollo web, el backend es la parte de la aplicación que reside en el servidor y que implementa toda la lógica de la aplicación para permitir que la aplicación trabaje con la base de datos}.
}

\newglossaryentry{librerías}{%
	name={librerías},
	description={En informática. una librería es un conjunto de implementaciones funcionales, codificadas en un lenguaje de programación, y que ofrece una interfaz bien definida para la funcionalidad que aporta}.
}

\newglossaryentry{SQL}{%
	name={SQL},
	description={SQL o Structured Query Language es un lenguaje de programación diseñado para administrar y recuperar información de una base datos, empleando el calculo relacional para hacerlo de una forma sencilla}.
}

\newglossaryentry{ORM}{%
	name={ORM},
	description={Un ORM o Object Relational Mapping es una técnica que nos permite interactuar con bases de datos relacionales empleando funciones u objetos en vez de consultas SQL. También se conoce como ORM al software o librería que implementa esta funcionalidad}.
}

\newglossaryentry{hashing}{%
	name={hasing},
	description={El hashing es una herramienta criptográfica que se emplea para generar un conjunto de datos cifrados a partir de otro conjunto de datos, como una cadena o un archivo.}.
}

\newglossaryentry{DOS}{%
	name={DOS},
	description={Un ataque DOS o Denial of Service es un tipo de ataque que consiste en saturar de la conexión de una aplicación o ordenador de forma que este no pueda ofrecer lo sevicios que estaba ofreciendo}.
}

\newglossaryentry{e2e}{%
	name={e2e},
	description={Las pruebas e2e (End-to-End) son un tipo de test que prueba el sistema como un conjunto, desde la perspectiva del usuario, probando todos los posibles caminos de la aplicación}.
}

\newglossaryentry{Markdown}{%
	name={Markdown},
	description={Markdown es un lenguaje de marcado ligero creado por John Gruber y Aaron Swartz cuya finalidad es conseguir la máxima legibilidad tanto en la entrada como en la salida}.
}

\newglossaryentry{ACID}{%
	name={ACID},
	description={En bases de datos se denomina ACID a las transacciones que son atómicas, consistentes, aisladas y durables}.
}

\newglossaryentry{Hosting}{%
	name={Hosting},
	description={Servicio de alojamiento web que permite publicar una página web o aplicación en internet}.
}





