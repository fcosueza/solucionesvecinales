% % % % % % % % % % % % % % % % % % % % % % % % % % % % % % % % % % % % % % % % % % % %
%                                                                                     %
% Short Sectioned Assignment LaTeX Template Version 1.0 (5/5/12)                      %
% This template has been downloaded from: http://www.LaTeXTemplates.com               %
%                                                                                     %
% Original author:  Frits Wenneker (http://www.howtotex.com)                          %
%                                                                                     %
% Modified by: Fco Javier Sueza Rodríguez (fcosueza@disroot.org)                      %
%                                                                                     %
% Changes:                                                                            %
%           - Document type scrarticle                                                %
%           - Use babel-lang-spanish package and marvosym                             %
%           - Use hyperref, enumitem and tcolorbox                                    %
%           - Use fancyhdr package an configure it                                    %
%           - Use Time New Roman (mathptmx), Helvetic and Courier fonts               %
%                                                                                     %
% License: CC BY-NC-SA 3.0 (http://creativecommons.org/licenses/by-nc-sa/3.0/)        %
%                                                                                     %
% % % % % % % % % % % % % % % % % % % % % % % % % % % % % % % % % % % % % % % % % % % %

%-----------------------------------------------%
%	              Packages                  %
%-----------------------------------------------%

\documentclass[bibliography=totoc]{scrarticle}

% ---- Text Input/Output ----- %

\usepackage[T1]{fontenc}
\usepackage[utf8]{inputenc}
\usepackage{mathptmx}
\usepackage[scaled=.92]{helvet}
\usepackage{courier}
\usepackage[indent=12pt]{parskip}

\usepackage{geometry}
\geometry{verbose,tmargin=3cm,bmargin=3cm,lmargin=2.5cm,rmargin=2.5cm}

% ---- Language ----- %

\usepackage[spanish]{babel}
\usepackage{marvosym}

% ---- Another packages ---- %

\usepackage{amsmath,amsfonts,amsthm}
\usepackage{graphics,graphicx}
\usepackage{tcolorbox}
\usepackage{hyperref}
\usepackage{enumitem}
\usepackage{fancyhdr}
\usepackage{float}

%--------------------------------------------------------------------%
%                      Customizing Document                          %
%--------------------------------------------------------------------%

% -------------- Customize headers and footers---------------------- %

\pagestyle{fancy}

\fancyhead[]{}
\fancyfoot[L]{}
\fancyfoot[C]{}
\fancyfoot[R]{\thepage}

\renewcommand{\headrulewidth}{0pt} % Remove header underlines
\renewcommand{\footrulewidth}{0pt} % Remove footer underlines

% Needed to get first page footer correctly
\fancypagestyle{plain}{
    \fancyhf{}
    \fancyfoot[R]{\thepage}
}

% --------------------- Packages Configuration --------------------- %


\hypersetup{
    colorlinks=true,
    linkcolor=black,
    urlcolor=magenta
}

\setcounter{secnumdepth}{4}
\setcounter{tocdepth}{4}
\graphicspath{{./images/}}

\setlength{\headheight}{13.6pt} % Customize the height of the header
\numberwithin{figure}{section} % Number figures within sections

% ------------------------ New Commands ----------------------------- %

\newcommand{\horrule}[1]{\rule{\linewidth}{#1}} % Create horizontal rule command


%----------------------------------------------------------------------------------------
%	TÍTULO Y DATOS DEL ALUMNO
%----------------------------------------------------------------------------------------

\title{
\vspace{10ex}
\normalfont \normalsize
\huge \textbf{Proyecto Web: SolucionesVecinales}
}
\author{Francisco Javier Sueza Rodríguez}
\date{\normalsize\today}

%----------------------------------------------------------------------------------------
%                                     DOCUMENTO
%----------------------------------------------------------------------------------------
\begin{document}


\maketitle

\thispagestyle{empty}

\vspace{80ex}

\begin{center}
    \begin{tabular}{l l}
        \textbf{Centro}: & IES Aguadulce \\
        \textbf{Ciclo Formativo}: & Desarrollo Aplicaciones Web (Distancia)\\
    \end{tabular}
\end{center}

\newpage

\tableofcontents

\newpage

\section{Introducción}

\subsection{Presentación}
En este documento se detalla el proyecto final desarrollado para el \textbf{CFGS Desarrollo de Aplicaciones Web} en el \textbf{IES Aguadulce}. El proyecto elegido ha sido una aplicación web para la \textbf{gestión y administración de comunidades de vecinos}, llamada \textbf{SolucionesVecinales}.

El objetivo de esta aplicación es facilitar la \textbf{gestión de comunidades de vecinos} por parte tanto de los administradores de la propiedad como de los inquilinos o propietarios, ofreciéndose de \textbf{forma gratuita} y poniendo especial énfasis en las \textbf{accesibilidad} y \textbf{facilidad de uso}, con una interfaz web amigable y accesible desde cual dispositivo que disponga una navegador web.

La aplicación trabaja con \textbf{3 tipos diferentes de usuarios}, incluyendo administradores, usuarios y usuarios no registrados, siendo el público objetivo los administradores de propiedad y presidentes de comunidad así como todos los inquilinos y propietarios de las comunidades de vecinos. Los diferentes usuarios tendrán diferentes permisos y responsabilidades.

A lo largo de este documento, desgranaremos todas los requisitos de la aplicación, el diseño elegido y todos los detalles del proceso de desarrollo y testeo.

\subsection{Contexto}
Esta aplicación es un \textbf{proyecto personal} que se ha decidido realizar tras el análisis de diferentes aplicaciones similares y ver determinadas carencias que pueden dificultar la inclusión de todos los potenciales usuarios. Hay que tener en cuenta, que los usuarios objetivos son un grupo muy heterogéneo donde podemos encontrar personas con diferente formación, conocimientos sobre tecnologías o con problemas de accesibilidad, así como gente con dificultades económicas.

Por este motivo, se ha decidido realizar una \textbf{aplicación gratuita}, centrándonos en la \textbf{accesibilidad} y \textbf{facilidad de uso}, que ayude a cualquier potencial usuario a usar la aplicación e implicarse en la gestión de comunidad de vecinos. 

\subsection{Planteamiento del Problema}
La gestión de comunidades de vecinos es una tarea compleja que requiere realizar un conjunto de gestiones administrativas asegurando la participación de los vecinos y de una forma transparente. En este aspecto, nuestra aplicación debería permitir que se realicen las siguientes tareas:

\begin{itemize}
	\item \textbf{Gestión de documentos}: gestión de los documentos generados en los diferentes procesos, como resolución de incidencias, documentos sobre juntas vecinales, etc. Permitiendo que estos puedan ser accedidos por todos los vecinos de la comunidad.
	
	\item \textbf{Reserva de espacios comunes}: automatizar la reserva y gestión de espacios comunes permitiendo a los vecinos reservar o cancelar reservas de dichos espacios, así como acceder a una lista con los espacios y el estado actual de reserva.
	
	\item \textbf{Gestión de Incidencias}: se debe permitir la creación de incidencias por parte de los vecinos así como su administración por parte del presidente de la comunidad o el administrador de fincas.
	
	\item \textbf{Gestión de Usuarios y Comunidades}: la aplicación debe permitir la creación, baja, actualización y consulta de usuarios con diferentes roles así como de comunidades de vecinos.
\end{itemize}

Además de estas tareas, hay que garantizar la seguridad de la información almacenada, ya que es información sensible, así como el proceso para realizar estas tareas sea sencillo e intuitivo, además de accesible. 

\subsection{Estructura de este Documento}
Este documento se compone de 6 secciones principales que tratan diferentes aspecto del proyecto. Para facilitar al interesado la navegación por éste, vamos a explicar brevemente en que consiste cada sección del documento:

\begin{itemize}
	\item \textbf{Análisis de Requisitos}: en esta sección se especifican los requisitos, tanto funcionales como no funcionales de la aplicación, especificando que es lo que se quiere conseguir con este proyecto y cuales son las restricciones que vienen impuestas. bien por el problema a tratar o por los propios requisitos.
	
	\item \textbf{Hardware y Software Necesario.}: en esta sección se especifica el hardware necesarios para realizar el desarrollo y poner en funcionamiento la aplicación así como el software que se va a necesitar para este mismo propósito. También se especifica el presupuesto y los costes de hosting para la aplicación.
	
	\item \textbf{Casos de Uso}: en esta sección se realiza un análisis de los principales casos de uso, empleando diferentes diagramas, así como un descripción más detalladas de los casos de uso generales.
	
	\item \textbf{Diseño de la Interfaz}: en esta sección se especifica el diseño de la interfaz de usuario, mostrando todas las páginas de la aplicación así como la relación entre estas y sus diferentes elementos.
	
	\item \textbf{Diseño de la Base de Datos}: en este apartado se especifica el diseño de la base de datos, empleando un esquema Entidad-Relación así como el paso a tablas de esta, explicando los puntos que sean oportunos sobre las decisiones de diseño tomadas. Además, se incluyen dos script, uno para la creación de la base de datos y otro para poblarla con datos.
	
	\item \textbf{Diagrama de Componentes}: en esta sección se muestra el diagrama de componentes de la aplicación, poniendo de relieve la arquitectura de la aplicación así como las interacción entre los diferentes elementos y sistemas.
\end{itemize}



\section{Análisis de Requisitos}

\subsection{Introducción}
La \textbf{administración de comunidades de vecinos} es una actividad compleja que requiere de la gestión de diferentes actividades, como la administración económica de la comunidad, la resolución de incidencias, gestión de los espacios comunitarios, resolución de incidencias o generación y comunicación de documentación, entre otras. 

Además, hay que \textbf{coordinar a los vecinos}, teniendo en cuenta que estos pueden ser un grupo muy heterogéneo, con diferentes dificultades, como \textbf{problemas de accesibilidad} o \textbf{movilidad}, \textbf{dificultad} para \textbf{manejar las nuevas tecnologías}, \textbf{problemas económicos}, etc. En resumen, una serie de circunstancias que añaden complejidad a una tarea ya de por si compleja.

Para solventar estos problemas, \textbf{SolucionesVecinales} ofrece una solución integral de gestión de comunidades de vecinos, enfocada en la simpleza de uso y la accesibilidad y totalmente gratuita, para no dejar a nadie fuera del proceso de gestión de su comunidad.





%\bibliographystyle{unsrt}

\end{document}