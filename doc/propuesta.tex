% % % % % % % % % % % % % % % % % % % % % % % % % % % % % % % % % % % % % % % % % % % %
%                                                                                     %
% Short Sectioned Assignment LaTeX Template Version 1.0 (5/5/12)                      %
% This template has been downloaded from: http://www.LaTeXTemplates.com               %
%                                                                                     %
% Original author:  Frits Wenneker (http://www.howtotex.com)                          %
%                                                                                     %
% Modified by: Fco Javier Sueza Rodríguez (fcosueza@disroot.org)                      %
%                                                                                     %
% Changes:                                                                            %
%           - Document type scrarticle                                                %
%           - Use babel-lang-spanish package and marvosym                             %
%           - Use hyperref, enumitem and tcolorbox                                    %
%           - Use fancyhdr package an configure it                                    %
%           - Use Time New Roman (mathptmx), Helvetic and Courier fonts               %
%                                                                                     %
% License: CC BY-NC-SA 3.0 (http://creativecommons.org/licenses/by-nc-sa/3.0/)        %
%                                                                                     %
% % % % % % % % % % % % % % % % % % % % % % % % % % % % % % % % % % % % % % % % % % % %

%-----------------------------------------------%
%	              Packages                  %
%-----------------------------------------------%

\documentclass[bibliography=totoc]{scrarticle}

% ---- Text Input/Output ----- %

\usepackage[T1]{fontenc}
\usepackage[utf8]{inputenc}
\usepackage{mathptmx}
\usepackage[scaled=.92]{helvet}
\usepackage{courier}
\usepackage[indent=12pt]{parskip}

\usepackage{geometry}
\geometry{verbose,tmargin=3cm,bmargin=3cm,lmargin=2.5cm,rmargin=2.5cm}

% ---- Language ----- %

\usepackage[spanish]{babel}
\usepackage{marvosym}

% ---- Another packages ---- %

\usepackage{amsmath,amsfonts,amsthm}
\usepackage{graphics,graphicx}
\usepackage{tcolorbox}
\usepackage{hyperref}
\usepackage{enumitem}
\usepackage{fancyhdr}
\usepackage{float}

%--------------------------------------------------------------------%
%                      Customizing Document                          %
%--------------------------------------------------------------------%

% -------------- Customize headers and footers---------------------- %

\pagestyle{fancy}

\fancyhead[]{}
\fancyfoot[L]{}
\fancyfoot[C]{}
\fancyfoot[R]{\thepage}

\renewcommand{\headrulewidth}{0pt} % Remove header underlines
\renewcommand{\footrulewidth}{0pt} % Remove footer underlines

% Needed to get first page footer correctly
\fancypagestyle{plain}{
    \fancyhf{}
    \fancyfoot[R]{\thepage}
}

% --------------------- Packages Configuration --------------------- %


\hypersetup{
    colorlinks=true,
    linkcolor=black,
    urlcolor=magenta
}

\setcounter{secnumdepth}{4}
\setcounter{tocdepth}{4}
\graphicspath{{./images/}}

\setlength{\headheight}{13.6pt} % Customize the height of the header
\numberwithin{figure}{section} % Number figures within sections

% ------------------------ New Commands ----------------------------- %

\newcommand{\horrule}[1]{\rule{\linewidth}{#1}} % Create horizontal rule command


%----------------------------------------------------------------------------------------
%	TÍTULO Y DATOS DEL ALUMNO
%----------------------------------------------------------------------------------------

\title{
\vspace{10ex}
\normalfont \normalsize
\huge \textbf{Proyecto Web: SolucionesVecinales}
}
\author{Francisco Javier Sueza Rodríguez}
\date{\normalsize\today}

\makeglossaries
\loadglsentries{glos}

%----------------------------------------------------------------------------------------
%                                     DOCUMENTO
%----------------------------------------------------------------------------------------
\begin{document}


\maketitle

\thispagestyle{empty}

\vspace{80ex}

\begin{center}
    \begin{tabular}{l l}
        \textbf{Centro}: & IES Aguadulce \\
        \textbf{Ciclo Formativo}: & Desarrollo Aplicaciones Web (Distancia)\\
    \end{tabular}
\end{center}

\newpage

\begingroup
\hypersetup{linkcolor=black}
\tableofcontents
\endgroup

\newpage

\section{Introducción}

\subsection{Presentación}
En este documento se detalla el proyecto final desarrollado para el \textbf{CFGS Desarrollo de Aplicaciones Web} en el \textbf{IES Aguadulce}. El proyecto elegido ha sido una aplicación web para la \textbf{gestión y administración de comunidades de vecinos}, llamada \textbf{SolucionesVecinales}.

El objetivo de esta aplicación es facilitar la \textbf{gestión de comunidades de vecinos} por parte tanto de los administradores de la propiedad como de los inquilinos o propietarios, ofreciéndose de \textbf{forma gratuita} y poniendo especial énfasis en las \textbf{accesibilidad} y \textbf{facilidad de uso}, con una interfaz web amigable y accesible desde cual dispositivo que disponga una navegador web.

La aplicación trabaja con \textbf{3 tipos diferentes de usuarios}, incluyendo administradores, usuarios y usuarios no registrados, siendo el público objetivo los administradores de propiedad y presidentes de comunidad así como todos los inquilinos y propietarios de las comunidades de vecinos. Los diferentes usuarios tendrán diferentes permisos y responsabilidades.

A lo largo de este documento, desgranaremos todas los requisitos de la aplicación, el diseño elegido y todos los detalles del proceso de desarrollo y testeo.

\subsection{Contexto}
Esta aplicación es un \textbf{proyecto personal} que se ha decidido realizar tras el análisis de diferentes aplicaciones similares y ver determinadas carencias que pueden dificultar la inclusión de todos los potenciales usuarios. Hay que tener en cuenta, que los usuarios objetivos son un grupo muy heterogéneo donde podemos encontrar personas con diferente formación, conocimientos sobre tecnologías o con problemas de accesibilidad, así como gente con dificultades económicas.

Por este motivo, se ha decidido realizar una \textbf{aplicación gratuita}, centrándonos en la \textbf{accesibilidad} y \textbf{facilidad de uso}, que ayude a cualquier potencial usuario a usar la aplicación e implicarse en la gestión de comunidad de vecinos. 

\subsection{Planteamiento del Problema}
La gestión de comunidades de vecinos es una tarea compleja que requiere realizar un conjunto de gestiones administrativas asegurando la participación de los vecinos y de una forma transparente. En este aspecto, nuestra aplicación debería permitir que se realicen las siguientes tareas:

\begin{itemize}
	\item \textbf{Gestión de documentos}: gestión de los documentos generados en los diferentes procesos, como resolución de incidencias, documentos sobre juntas vecinales, etc. Permitiendo que estos puedan ser accedidos por todos los vecinos de la comunidad.
	
	\item \textbf{Reserva de espacios comunes}: automatizar la reserva y gestión de espacios comunes permitiendo a los vecinos reservar o cancelar reservas de dichos espacios, así como acceder a una lista con los espacios y el estado actual de reserva.
	
	\item \textbf{Gestión de Incidencias}: se debe permitir la creación de incidencias por parte de los vecinos así como su administración por parte del presidente de la comunidad o el administrador de fincas.
	
	\item \textbf{Gestión de Usuarios y Comunidades}: la aplicación debe permitir la creación, baja, actualización y consulta de usuarios con diferentes roles así como de comunidades de vecinos.
	
	\item \textbf{Gestión de Finanzas}: la aplicación debe realizar, de forma automática, cálculos para administrar económicamente la comunidad, calculando el balance anual a partir de las cuotas que se están pagando y las diferentes facturas que se han pagado.
\end{itemize}

Además de estas tareas, hay que garantizar la seguridad de la información almacenada, ya que es información sensible, así como el proceso para realizar estas tareas sea sencillo e intuitivo, además de accesible. 

\subsection{Visión General}
Este documento se compone de 6 secciones principales que tratan diferentes aspecto del proyecto. Para facilitar al interesado la navegación por éste, vamos a explicar brevemente en que consiste cada sección del documento:

\begin{itemize}
	\item \textbf{Análisis de Requisitos}: en esta sección se especifican los requisitos, tanto funcionales como no funcionales de la aplicación, especificando que es lo que se quiere conseguir con este proyecto y cuales son las restricciones que vienen impuestas. bien por el problema a tratar o por los propios requisitos.
	
	\item \textbf{Hardware y Software Necesario.}: en esta sección se especifica el hardware necesarios para realizar el desarrollo y poner en funcionamiento la aplicación así como el software que se va a necesitar para este mismo propósito. También se especifica el presupuesto y los costes de hosting para la aplicación.
	
	\item \textbf{Casos de Uso}: en esta sección se realiza un análisis de los principales casos de uso, empleando diferentes diagramas, así como un descripción más detalladas de los casos de uso generales.
	
	\item \textbf{Diseño de la Interfaz}: en esta sección se especifica el diseño de la interfaz de usuario, mostrando todas las páginas de la aplicación así como la relación entre estas y sus diferentes elementos.
	
	\item \textbf{Diseño de la Base de Datos}: en este apartado se especifica el diseño de la base de datos, empleando un esquema Entidad-Relación así como el paso a tablas de esta, explicando los puntos que sean oportunos sobre las decisiones de diseño tomadas. Además, se incluyen dos script, uno para la creación de la base de datos y otro para poblarla con datos.
	
	\item \textbf{Diagrama de Componentes}: en esta sección se muestra el diagrama de componentes de la aplicación, poniendo de relieve la arquitectura de la aplicación así como las interacción entre los diferentes elementos y sistemas.
\end{itemize}



\section{Análisis de Requisitos}

\subsection{Introducción}
En esta sección se va definir la \textbf{especificación de requerimientos} que establece los requisitos funcionales y no funcionales de la aplicación SolucionesVecinales, realizando una análisis del funcionamiento de esperado de aplicación y estableciendo unos limites para el diseño de ésta.

\subsubsection{Propósito}
El propósito de esta sección es \textbf{establecer los requisitos}, tanto funcionales como no funcionales, de la aplicación que se va a desarrollar, realizando una \textbf{análisis exhaustivo de su funcionamiento} y estableciendo las restricciones necesarias para que la aplicación cumpla su funcionalidad de forma correcta, segura y eficiente. Estos requisitos no ayudarán a planificar el proceso de desarrollo así como ha tomar las decisiones oportunas durante el proceso de diseño de la aplicación.

Esta sección va dirigida principalmente al equipo de desarrollo, que serán los encargados de elaborar el diseño de la aplicación a partir de los requerimientos y restricciones establecidos en este documento, aunque teniendo en cuenta que también esta incluido en el documento del proyecto en general cabe la posibilidad de que tengan acceso a el otros interesados, como por ejemplo, posibles inversores.

\subsubsection{Alcance}
Nuestro producto se llama \textbf{SolucionesVecinales} y es una aplicación web para la \textbf{gestión de comunidades de vecinos}. El fin de SolucionesVecinales es ayudar a todos los integrantes en una comunidad de vecinos a realizar diferentes gestiones relacionadas con las administración de su comunidad, de forma simple y sencilla. 

Es una aplicación que se va ha \textbf{ofrece de forma gratuita}, cuya meta principal es llegar al mayor número de usuarios, implementando buenas \textbf{políticas de accesibilidad} y ofreciendo una \textbf{interfaz sencilla} e \textbf{intuitiva} que pueda usar cualquiera persona, incluso aquellos que no están familiarizados con las nuevas tecnologías. 

Partiendo de esta base, la aplicación deberá \textbf{ayudar a realizar las tareas} más comunes en las que deben participar los vecinos en una comunidad, como el acceso de documentos, creación de incidencias, consulta de las cuentas de la comunidad,  reserva de espacios comunes, etc.

\subsubsection{Definiciones, Siglas y Abreviaturas}
Las definiciones, siglas y abreviaturas se incluyen al final del documento, dentro de un Glosario, donde se incluyen todos los términos tanto de esta sección como del resto de secciones.

\subsubsection{Referencias}
La referencias, al igual que el glosario, tanto de esta sección como del resto de secciones se incluyen al final de documento, para no duplicar secciones. Además, en cada texto que haga referencia a alguna página o documento, se incluirá la cita adecuada apunta a dicha referencia, para facilitar su acceso.


\subsubsection{Visión General}
Este documento se divide principalmente en 4 secciones, incluyendo la actual. Cada sección describe un aspecto diferentes de los requisitos del software y son las siguientes:

\begin{itemize}
	\item \textbf{Introducción} (Sección 2.1): en esta primera sección se realiza una introducción tanto al software como a este documento, explicando su propósito, alcance, etc...
	
	\item \textbf{Descripción General} (Sección 2.2): en esta sección se realiza una descripción del software que se va a desarrollar, incluyendo su funcionalidad, restricciones, características de los usuarios potenciales, interfaces, etc. Nos sirve como base para especificar los requisitos en el siguiente apartado.
	
	\item \textbf{Requerimientos Específicos} (Sección 2.3): en la tercera sección vamos a establecer los requisitos específicos del software, extraídos de la descripción general y separándolos en requisitos funcionales y no funcionales.
	
	\item \textbf{Requerimientos de Documentación} (Sección 2.4): en esta sección de la parte de requisitos se especifican los requisitos de documentación que va a tener el software, incluyendo manuales, ayuda en línea, instalación, etc.
\end{itemize}

\subsection{Descripción General}
En esta sección se van a describir todos los factores y restricciones que afectan al producto, estableciendo una base para la definición de los requisitos específicos en la siguiente sección, incluyendo una perspectiva general del producto, así como las restricciones que se deben aplicar en su diseño.

\subsubsection{Perspectiva del Producto}
En primer lugar vamos a realizar una descripción del producto, especialmente de las diferentes interfaces que se van a emplear y su interacción con los diferentes elementos y sistemas con los que deberá interactuar nuestro software.

\paragraph{Interfaz de Usuario}
~\\\\
\-\ \-\ \-\ Debido a que las especificaciones sobre al interfaz de usuario son más extensas, se han incluido como un anexo. Estas especificaciones se pueden consultar, en concreto, en el \hyperref[sec:apenA]{Anexo A}.

\newpage

\appendix

\begin{appendices}
	
\section{Descripción de la Interfaz de Usuario}
\label{sec:apenA}

La interfaz de la aplicación es una \textbf{interfaz web}, a la que se accederá desde un navegador. Esta estará compuesta de varias pantallas que ayudarán al usuario a navegar por la aplicación y llevar a cabo las diferentes tareas y funciones que se pueden realizar en nuestro software.

Las interfaces de las que estará compuesta nuestra aplicación son las siguientes:

\begin{itemize}
	\item \textbf{Pantalla Principal}: esta es la primera interfaz con la que se encuentra un usuario al ingresar en nuestra aplicación web. La función principal de esta pantalla es la de \textbf{ofrecer información sobre la aplicación}, así como proporcionar un\textbf{ menú con diferentes opciones} que, además de permitir la navegación por las diferentes secciones permita a un usuario \textbf{registrarse} o \textbf{realizar \gls{login}}.
	
	Las secciones de las que estará compuesta esta interfaz serán las siguientes:
	
	\begin{itemize}
		\item \textbf{Cabecera}: aquí se mostrará una imagen con un eslogan, el nombre de la aplicación y un elemento \gls{CTO} para que los usuarios se registren o realicen login.
		
		\item \textbf{Barra de Menú}: esta barra, que se sitúa inicialmente encima de la cabecera y que debe permanecer en todo momento en la parte superior de la pantalla, esta compuesto del \textbf{logo de la aplicación} en el lado derecho y de un \textbf{menú}, con dos partes en el lado izquierdo. La primera parte contendrá las diferentes secciones de esta página para que se puedan acceder a ellas rápidamente. Separada un poco encontraremos la opción de login, para permitir a los usuarios realizar login. 
		
		\item \textbf{Secciones}: esta página contendrá, como mínimo, 3 secciones, donde se explicará un poco el propósito del software, sus principales características, los que nos diferencia de la competencia, etc. El número de de secciones puede variar, pudiendo ser necesario agregar más, pero estás deben ser visualmente atractivas, incluyendo imágenes e iconos y con una disposición adecuada. Una sección que deberá ser obligatoria, siendo la última de éstas, es la sección "Contacta con nosotros", donde ser proveerá un formulario para que los usuarios puedan realizar cualquier consulta que necesiten.
		
		\item \textbf{Pie de Página}: en el pie de página se añadirá una lista de enlaces con las diferentes secciones de la página, un mapa del sitio web y un conjunto de enlaces de redes sociales de la aplicación o en su defecto del equipo de desarrollo.
	\end{itemize}
	
	\item \textbf{Pantalla de Login}: a esta pantalla se accede desde la pantalla principal cuando el usuario pulsa en el enlace de login/registro del menú de la barra principal. La pantalla inicial esta compuesta por un \textbf{formulario} de login, que debe aceptar un email y una contraseña como entrada con un botón enviar. Además, incluirá un enlace debajo del botón, centrado, que permitirá al usuario registrarse, en caso de que no este registrado, y que desplegará un nuevo formulario que veremos en detalle a continuación.
	
	\item \textbf{Pantalla de Registro}: en esta pantalla, que se accede desde la pantalla de login, se muestra un formulario que permitirá a un usuario registrarse en el sistema. Este formulario estará compuesto por diferentes campos que permitirán al usuario introducir la información necesaria pare registrar, como puede ser Nombre, Correo, Dirección, rol, etc. Debajo, aparecerá el \textbf{botón registrarse} que permitirá al usuario registrar todos sus datos. 
	
	\item \textbf{Pantalla de Comunidad}: una vez registrado o realizado el login, al usuario se le mostrará la página  principal de login. Esta página variará dependiendo del estado en el que se encuentre el usuario y el rol que tenga, aunque tendrá 4 zonas definidas:
	
	\begin{itemize}
		\item \textbf{Barra de Menú}: similar a la que vemos en la página principal, pero las opciones de menú "Login/Registrarse" habrá cambiado por el \textbf{icono de un usuario}, mostrando un \textbf{menú vertical} con las opciones "\textbf{Perfil}" y "\textbf{Logout}", que permitirán accedes a la información de perfil y hacer \gls{logout} respectivamente.
		
		\item \textbf{Menú Lateral}: un menú latera, situado a la izquierda, que mostrará las opciones relativas a la comunidad que este cargada. Las opciones variarán según si el usuario tiene el rol de administrador o de usuario, pero la opciones comunes a ambos son las siguientes: 
		
		\begin{itemize}
			\item \textbf{Tablón de Anuncios}
			\item \textbf{Documentos}
			\item \textbf{Incidencias}
		\end{itemize}
	\end{itemize} 
	

	
	
	\item \textbf{Pantalla de Perfil}: en este pantalla, que se accede desde el menú vertical del icono de perfil, se mostrará un \textbf{formulario con toda la información del usuario}. Este formulario será \textbf{editable} y se podrá actualizar mediante un botón que habrá debajo de él con el texto "Guardar". 
	
	Además, debajo de este aparecerá un \textbf{botón} con la opción "\textbf{Eliminar usuario}" que permitirá a un usuario eliminar su información, y que al ser pulsado ocultará el formulario con al información y mostrará una ventana de confirmación. Si la confirmación es positiva, se eliminará al usuario, y se redireccionará la pantalla principal.
	
	\item \textbf{Pantalla de Añadir Comunidad}: esta pantalla, a la que solo tienen acceso los usuarios con rol administrador, permite \textbf{añadir una nueva comunidad} de vecinos. La interfaz mostrará un \textbf{formulario web} con diferentes datos que deberán rellenarse para dar de alta la comunidad, principalmente la \textbf{dirección}, que será por lo que se identifique cada comunidad. En el caso de que se pueda dar de alta la comunidad se mostrará una mensaje de éxito y se redireccionará a la páginas principal de comunidad. En caso contrario, se mostrará los mensajes de error adecuados.
	
	\item \textbf{Pantalla del Tablón de Anuncios}: en esta interfaz se mostrará el menú lateral que corresponda al usuario y en la zona de contenido
	
	\item \textbf{}
	

		
	
	
\end{itemize}

\end{appendices}

\newpage

\printglossary	

\newpage

\bibliography{citas}
\bibliographystyle{unsrt}

\end{document}