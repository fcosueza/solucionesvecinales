% % % % % % % % % % % % % % % % % % % % % % % % % % % % % % % % % % % % % % % % % % % %
%                                                                                     %
% Short Sectioned Assignment LaTeX Template Version 1.0 (5/5/12)                      %
% This template has been downloaded from: http://www.LaTeXTemplates.com               %
%                                                                                     %
% Original author:  Frits Wenneker (http://www.howtotex.com)                          %
%                                                                                     %
% Modified by: Fco Javier Sueza Rodríguez (fcosueza@disroot.org)                      %
%                                                                                     %
% Changes:                                                                            %
%           - Document type scrarticle                                                %
%           - Use babel-lang-spanish package and marvosym                             %
%           - Use hyperref, enumitem and tcolorbox                                    %
%           - Use fancyhdr package an configure it                                    %
%           - Use Time New Roman (mathptmx), Helvetic and Courier fonts               %
%                                                                                     %
% License: CC BY-NC-SA 3.0 (http://creativecommons.org/licenses/by-nc-sa/3.0/)        %
%                                                                                     %
% % % % % % % % % % % % % % % % % % % % % % % % % % % % % % % % % % % % % % % % % % % %

%-----------------------------------------------%
%	              Packages                  %
%-----------------------------------------------%

\documentclass[bibliography=totoc]{scrarticle}

% ---- Text Input/Output ----- %

\usepackage[T1]{fontenc}
\usepackage[utf8]{inputenc}
\usepackage{mathptmx}
\usepackage[scaled=.92]{helvet}
\usepackage{courier}
\usepackage[indent=12pt]{parskip}

\usepackage{geometry}
\geometry{verbose,tmargin=3cm,bmargin=3cm,lmargin=2.5cm,rmargin=2.5cm}

% ---- Language ----- %

\usepackage[spanish]{babel}
\usepackage{marvosym}

% ---- Another packages ---- %

\usepackage{amsmath,amsfonts,amsthm}
\usepackage{graphics,graphicx}
\usepackage{tcolorbox}
\usepackage{hyperref}
\usepackage{enumitem}
\usepackage{fancyhdr}
\usepackage{float}

%--------------------------------------------------------------------%
%                      Customizing Document                          %
%--------------------------------------------------------------------%

% -------------- Customize headers and footers---------------------- %

\pagestyle{fancy}

\fancyhead[]{}
\fancyfoot[L]{}
\fancyfoot[C]{}
\fancyfoot[R]{\thepage}

\renewcommand{\headrulewidth}{0pt} % Remove header underlines
\renewcommand{\footrulewidth}{0pt} % Remove footer underlines

% Needed to get first page footer correctly
\fancypagestyle{plain}{
    \fancyhf{}
    \fancyfoot[R]{\thepage}
}

% --------------------- Packages Configuration --------------------- %


\hypersetup{
    colorlinks=true,
    linkcolor=black,
    urlcolor=magenta
}

\setcounter{secnumdepth}{4}
\setcounter{tocdepth}{4}
\graphicspath{{./images/}}

\setlength{\headheight}{13.6pt} % Customize the height of the header
\numberwithin{figure}{section} % Number figures within sections

% ------------------------ New Commands ----------------------------- %

\newcommand{\horrule}[1]{\rule{\linewidth}{#1}} % Create horizontal rule command


%----------------------------------------------------------------------------------------
%	TÍTULO Y DATOS DEL ALUMNO
%----------------------------------------------------------------------------------------

\title{
\vspace{10ex}
\normalfont \normalsize
\huge \textbf{Proyecto Web: SolucionesVecinales}
}
\author{Francisco Javier Sueza Rodríguez}
\date{\normalsize\today}

%----------------------------------------------------------------------------------------
%                                     DOCUMENTO
%----------------------------------------------------------------------------------------
\begin{document}


\maketitle

\thispagestyle{empty}

\vspace{80ex}

\begin{center}
    \begin{tabular}{l l}
        \textbf{Centro}: & IES Aguadulce \\
        \textbf{Ciclo Formativo}: & Desarrollo Aplicaciones Web (Distancia)\\
    \end{tabular}
\end{center}

\newpage

\tableofcontents

\newpage

\section{Introducción}

\subsection{Presentación}
En este documento se detalla el proyecto final desarrollado para el \textbf{CFGS Desarrollo de Aplicaciones Web} en el \textbf{IES Aguadulce}. El proyecto elegido ha sido una aplicación web para la \textbf{gestión y administración de comunidades de vecinos}, llamada \textbf{SolucionesVecinales}.

El objetivo de esta aplicación es facilitar la \textbf{gestión de comunidades de vecinos} por parte tanto de los administradores de la propiedad como de los inquilinos o propietarios, ofreciéndose de \textbf{forma gratuita} y poniendo especial énfasis en las \textbf{accesibilidad} y \textbf{facilidad de uso}, con una interfaz web amigable y accesible desde cual dispositivo que disponga una navegador web.

La aplicación trabaja con \textbf{3 tipos diferentes de usuarios}, incluyendo administradores, usuarios y usuarios no registrados, siendo el público objetivo los administradores de propiedad y presidentes de comunidad así como todos los inquilinos y propietarios de las comunidades de vecinos. Los diferentes usuarios tendrán diferentes permisos y responsabilidades.

A lo largo de este documento, desgranaremos todas los requisitos de la aplicación, el diseño elegido y todos los detalles del proceso de desarrollo y testeo.

\subsection{Contexto}
Esta aplicación es un \textbf{proyecto personal} que se ha decidido realizar tras el análisis de diferentes aplicaciones similares y ver determinadas carencias que pueden dificultar la inclusión de todos los potenciales usuarios. Hay que tener en cuenta, que los usuarios objetivos son un grupo muy heterogéneo donde podemos encontrar personas con diferente formación, conocimientos sobre tecnologías o con problemas de accesibilidad, así como gente con dificultades económicas.

Por este motivo, se ha decidido realizar una \textbf{aplicación gratuita}, centrándonos en la \textbf{accesibilidad} y \textbf{facilidad de uso}, que ayude a cualquier potencial usuario a usar la aplicación e implicarse en la gestión de comunidad de vecinos. 

\subsection{Planteamiento del Problema}




\section{Especificación de Requerimientos}

\subsection{Introducción}
La \textbf{administración de comunidades de vecinos} es una actividad compleja que requiere de la gestión de diferentes actividades, como la administración económica de la comunidad, la resolución de incidencias, gestión de los espacios comunitarios, resolución de incidencias o generación y comunicación de documentación, entre otras. 

Además, hay que \textbf{coordinar a los vecinos}, teniendo en cuenta que estos pueden ser un grupo muy heterogéneo, con diferentes dificultades, como \textbf{problemas de accesibilidad} o \textbf{movilidad}, \textbf{dificultad} para \textbf{manejar las nuevas tecnologías}, \textbf{problemas económicos}, etc. En resumen, una serie de circunstancias que añaden complejidad a una tarea ya de por si compleja.

Para solventar estos problemas, \textbf{SolucionesVecinales} ofrece una solución integral de gestión de comunidades de vecinos, enfocada en la simpleza de uso y la accesibilidad y totalmente gratuita, para no dejar a nadie fuera del proceso de gestión de su comunidad.





%\bibliographystyle{unsrt}

\end{document}