% % % % % % % % % % % % % % % % % % % % % % % % % % % % % % % % % % % % % % % % % % % %
%                                                                                     %
% Short Sectioned Assignment LaTeX Template Version 1.0 (5/5/12)                      %
% This template has been downloaded from: http://www.LaTeXTemplates.com               %
%                                                                                     %
% Original author:  Frits Wenneker (http://www.howtotex.com)                          %
%                                                                                     %
% Modified by: Fco Javier Sueza Rodríguez (fcosueza@disroot.org)                      %
%                                                                                     %
% Changes:                                                                            %
%           - Document type scrarticle                                                %
%           - Use babel-lang-spanish package and marvosym                             %
%           - Use hyperref, enumitem and tcolorbox                                    %
%           - Use fancyhdr package an configure it                                    %
%           - Use Time New Roman (mathptmx), Helvetic and Courier fonts               %
%                                                                                     %
% License: CC BY-NC-SA 3.0 (http://creativecommons.org/licenses/by-nc-sa/3.0/)        %
%                                                                                     %
% % % % % % % % % % % % % % % % % % % % % % % % % % % % % % % % % % % % % % % % % % % %

%-----------------------------------------------%
%	              Packages                  %
%-----------------------------------------------%

\documentclass[bibliography=totoc]{scrarticle}

% ---- Text Input/Output ----- %

\usepackage[T1]{fontenc}
\usepackage[utf8]{inputenc}
\usepackage{mathptmx}
\usepackage[scaled=.92]{helvet}
\usepackage{courier}
\usepackage[indent=12pt]{parskip}

\usepackage{geometry}
\geometry{verbose,tmargin=3cm,bmargin=3cm,lmargin=2.5cm,rmargin=2.5cm}

% ---- Language ----- %

\usepackage[spanish]{babel}
\usepackage{marvosym}

% ---- Another packages ---- %

\usepackage{amsmath,amsfonts,amsthm}
\usepackage{graphics,graphicx}
\usepackage{tcolorbox}
\usepackage{hyperref}
\usepackage{enumitem}
\usepackage{fancyhdr}
\usepackage{float}

%--------------------------------------------------------------------%
%                      Customizing Document                          %
%--------------------------------------------------------------------%

% -------------- Customize headers and footers---------------------- %

\pagestyle{fancy}

\fancyhead[]{}
\fancyfoot[L]{}
\fancyfoot[C]{}
\fancyfoot[R]{\thepage}

\renewcommand{\headrulewidth}{0pt} % Remove header underlines
\renewcommand{\footrulewidth}{0pt} % Remove footer underlines

% Needed to get first page footer correctly
\fancypagestyle{plain}{
    \fancyhf{}
    \fancyfoot[R]{\thepage}
}

% --------------------- Packages Configuration --------------------- %


\hypersetup{
    colorlinks=true,
    linkcolor=black,
    urlcolor=magenta
}

\setcounter{secnumdepth}{4}
\setcounter{tocdepth}{4}
\graphicspath{{./images/}}

\setlength{\headheight}{13.6pt} % Customize the height of the header
\numberwithin{figure}{section} % Number figures within sections

% ------------------------ New Commands ----------------------------- %

\newcommand{\horrule}[1]{\rule{\linewidth}{#1}} % Create horizontal rule command


%----------------------------------------------------------------------------------------
%	TÍTULO Y DATOS DEL ALUMNO
%----------------------------------------------------------------------------------------

\title{
\vspace{10ex}
\normalfont \normalsize
\huge \textbf{Proyecto Web: SolucionesVecinales}
}
\author{Francisco Javier Sueza Rodríguez}
\date{\normalsize\today}

%----------------------------------------------------------------------------------------
%                                     DOCUMENTO
%----------------------------------------------------------------------------------------
\begin{document}


\maketitle

\thispagestyle{empty}

\vspace{80ex}

\begin{center}
    \begin{tabular}{l l}
        \textbf{Centro}: & IES Aguadulce \\
        \textbf{Ciclo Formativo}: & Desarrollo Aplicaciones Web (Distancia)\\
    \end{tabular}
\end{center}

\newpage

\tableofcontents

\newpage

\section{Introducción}

\subsection{Presentación}
En este documento se detalla el proyecto final desarrollado para el \textbf{CFGS Desarrollo de Aplicaciones Web} en el \textbf{IES Aguadulce}. El proyecto elegido ha sido una aplicación web para la \textbf{gestión y administración de comunidades de vecinos}, llamada \textbf{SolucionesVecinales}.

El objetivo de esta aplicación es facilitar la \textbf{gestión de comunidades de vecinos} por parte tanto de los administradores de la propiedad como de los inquilinos o propietarios, ofreciéndose de \textbf{forma gratuita} y poniendo especial énfasis en las \textbf{accesibilidad} y \textbf{facilidad de uso}, con una interfaz web amigable y accesible desde cual dispositivo que disponga una navegador web.

La aplicación trabaja con \textbf{3 tipos diferentes de usuarios}, incluyendo administradores, usuarios y usuarios no registrados, siendo el público objetivo los administradores de propiedad y presidentes de comunidad así como todos los inquilinos y propietarios de las comunidades de vecinos. Los diferentes usuarios tendrán diferentes permisos y responsabilidades.

A lo largo de este documento, desgranaremos todas los requisitos de la aplicación, el diseño elegido y todos los detalles del proceso de desarrollo y testeo.

\subsection{Contexto}
Esta aplicación es un \textbf{proyecto personal} que se ha decidido realizar tras el análisis de diferentes aplicaciones similares y ver determinadas carencias que pueden dificultar la inclusión de todos los potenciales usuarios. Hay que tener en cuenta, que los usuarios objetivos son un grupo muy heterogéneo donde podemos encontrar personas con diferente formación, conocimientos sobre tecnologías o con problemas de accesibilidad, así como gente con dificultades económicas.

Por este motivo, se ha decidido realizar una \textbf{aplicación gratuita}, centrándonos en la \textbf{accesibilidad} y \textbf{facilidad de uso}, que ayude a cualquier potencial usuario a usar la aplicación e implicarse en la gestión de comunidad de vecinos. 

\subsection{Planteamiento del Problema}
La gestión de comunidades de vecinos es una tarea compleja que requiere realizar un conjunto de gestiones administrativas asegurando la participación de los vecinos y de una forma transparente. En este aspecto, nuestra aplicación debería permitir que se realicen las siguientes tareas:

\begin{itemize}
	\item \textbf{Gestión de documentos}: gestión de los documentos generados en los diferentes procesos, como resolución de incidencias, documentos sobre juntas vecinales, etc. Permitiendo que estos puedan ser accedidos por todos los vecinos de la comunidad.
	
	\item \textbf{Reserva de espacios comunes}: automatizar la reserva y gestión de espacios comunes permitiendo a los vecinos reservar o cancelar reservas de dichos espacios, así como acceder a una lista con los espacios y el estado actual de reserva.
	
	\item \textbf{Gestión de Incidencias}: se debe permitir la creación de incidencias por parte de los vecinos así como su administración por parte del presidente de la comunidad o el administrador de fincas.
	
	\item \textbf{Gestión de Usuarios y Comunidades}: la aplicación debe permitir la creación, baja, actualización y consulta de usuarios con diferentes roles así como de comunidades de vecinos.
\end{itemize}

Además de estas tareas, hay que garantizar la seguridad de la información almacenada, ya que es información sensible, así como el proceso para realizar estas tareas sea sencillo e intuitivo, además de accesible. 

\subsection{Estructura de este Documento}
Este documento se compone de 6 secciones principales que tratan diferentes aspecto del proyecto. Para facilitar al interesado la navegación por éste, vamos a explicar brevemente en que consiste cada sección del documento:

\begin{itemize}
	\item \textbf{Análisis de Requisitos}: en esta sección se especifican los requisitos, tanto funcionales como no funcionales de la aplicación, especificando que es lo que se quiere conseguir con este proyecto y cuales son las restricciones que vienen impuestas. bien por el problema a tratar o por los propios requisitos.
	
	\item \textbf{Hardware y Software Necesario.}: en esta sección se especifica el hardware necesarios para realizar el desarrollo y poner en funcionamiento la aplicación así como el software que se va a necesitar para este mismo propósito. También se especifica el presupuesto y los costes de hosting para la aplicación.
	
	\item \textbf{Casos de Uso}: en esta sección se realiza un análisis de los principales casos de uso, empleando diferentes diagramas, así como un descripción más detalladas de los casos de uso generales.
	
	\item \textbf{Diseño de la Interfaz}: en esta sección se especifica el diseño de la interfaz de usuario, mostrando todas las páginas de la aplicación así como la relación entre estas y sus diferentes elementos.
	
	\item \textbf{Diseño de la Base de Datos}: en este apartado se especifica el diseño de la base de datos, empleando un esquema Entidad-Relación así como el paso a tablas de esta, explicando los puntos que sean oportunos sobre las decisiones de diseño tomadas. Además, se incluyen dos script, uno para la creación de la base de datos y otro para poblarla con datos.
	
	\item \textbf{Diagrama de Componentes}: en esta sección se muestra el diagrama de componentes de la aplicación, poniendo de relieve la arquitectura de la aplicación así como las interacción entre los diferentes elementos y sistemas.
\end{itemize}



\section{Análisis de Requisitos}

\subsection{Introducción}
En esta sección se va definir la \textbf{especificación de requerimientos} que establece los requisitos funcionales y no funcionales de la aplicación SolucionesVecinales, realizando una análisis del funcionamiento de esperado de aplicación y estableciendo unos limites para el diseño de ésta.

\subsubsection{Propósito}
El propósito de esta sección es \textbf{establecer los requisitos}, tanto funcionales como no funcionales, de la aplicación que se va a desarrollar, realizando una \textbf{análisis exhaustivo de su funcionamiento} y estableciendo las restricciones necesarias para que la aplicación cumpla su funcionalidad de forma correcta, segura y eficiente. Estos requisitos no ayudarán a planificar el proceso de desarrollo así como ha tomar las decisiones oportunas durante el proceso de diseño de la aplicación.

Esta sección va dirigido principalmente al equipo de desarrollo, que serán los encargados de elaborar el diseño de la aplicación a partir de los requerimientos y restricciones establecidos en este documento, aunque teniendo en cuenta que también esta incluido en el documento del proyecto en general cabe la posibilidad de que tengan acceso a el otros interesados, como por ejemplo, posibles inversores.

\subsubsection{Alcance}
Nuestro producto se llama \textbf{SolucionesVecinales} y es una aplicación web para la \textbf{gestión de comunidades de vecinos}. El fin de SolucionesVecinales es ayudar a todos los integrantes en una comunidad de vecinos a realizar diferentes gestiones relacionadas con las administración de su comunidad, de forma simple y sencilla. 

Es una aplicación que se va ha \textbf{ofrece de forma gratuita}, cuya meta principal es llegar al mayor número de usuarios, implementando buenas \textbf{políticas de accesibilidad} y ofreciendo una \textbf{interfaz sencilla} e \textbf{intuitiva} que pueda usar cualquiera persona, incluso aquellos que no están familiarizados con las nuevas tecnologías. 

Partiendo de esta base, la aplicación deberá \textbf{ayudar a realizar las tareas} más comunes en las que deben participar los vecinos en una comunidad, como el acceso de documentos, creación de incidencias, consulta de las cuentas de la comunidad,  reserva de espacios comunes, etc.

\subsubsection{Definiciones, Siglas y Abreviaturas}
En esta sección se va a realizar una descripción de las definiciones, siglas o abreviaturas que puedan aparecer a lo largo de este documento y que sea oportuno explicar con más detalle.
 
\begin{description}
		\item[API:] una API del inglés Application Programming Interface ó Interfaz de Programación de Aplicaciones, es un conjunto de definiciones, protocolos, funciones, etc.., que permiten que una aplicación de comunique con otra aplicación o pieza de software. \cite{aws01}
		
		\item[API Web:] una API Web es un tipo de API que permite la comunicación entre el navegador (cliente) y el servidor, permitiendo que el navegador realice diferentes peticiones HTTP al servidor que permiten modificar, añadir, consultar o eliminar los datos almacenados en el servidor y el acceso a diferentes servicios y recursos.
		
		\item[API REST:] este tipo de API es uno de los más empleados en las aplicaciones web actualmente. El termino REST proviene del ingles Representational State Transfer, ó transferencia de estado representacional. Estás APIs definen un conjunto de funciones, como GET, PUT, DElETE, etc, que los clientes pueden utilizar para acceder a los datos del servidor. Su principal características es que no tiene estado, es decir, los servidores no guardan los datos del cliente entre peticiones. \cite{wiki01}
\end{description}

\subsubsection{Referencias}
La referencias, tanto de esta sección como del resto de secciones se incluyen al final de documento, con los enlaces adecuados en los textos que hacen referencias a estas.


\subsection{Visión General}








\newpage

\bibliography{citas}
\bibliographystyle{unsrt}

\end{document}